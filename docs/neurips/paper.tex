% Created 2020-05-27 Wed 15:23
% Intended LaTeX compiler: pdflatex
\documentclass{article}
                                  \usepackage[nonatbib]{neurips_2020}
\usepackage[utf8]{inputenc}
\usepackage[T1]{fontenc}
\usepackage[table,svgnames]{xcolor}
\PassOptionsToPackage{hyphens}{url}\usepackage[hidelinks]{hyperref}
\usepackage{microtype}
\usepackage{graphicx}
\usepackage{amsmath}
\usepackage{amssymb}
\usepackage{multirow}
\usepackage{mathtools}
\usepackage{nicematrix}
\usepackage{cleveref}
\usepackage{txfonts}
\renewcommand{\vec}[1]{\bm{#1}}
\newcommand{\Set}[1]{\mathcal{#1}}
\newcommand{\todo}[1]{{\color{red}TODO: #1}}
\author{Maximilian Nickel}
\date{}
\title{Latent Coupling for High-Resolution \\  COVID-19 Spread Forecasting}
\hypersetup{
 pdfauthor={Maximilian Nickel},
 pdftitle={Latent Coupling for High-Resolution \\  COVID-19 Spread Forecasting},
 pdfkeywords={},
 pdfsubject={},
 pdfcreator={Emacs 26.3 (Org mode 9.4)}, 
 pdflang={English}}
\begin{document}

\maketitle
\author{}


\begin{abstract}
Modeling the spread of COVID-19 at a high spatial and temporal resolution has
become a key task in the public health response to the disease. This poses
unique challenges due to the novelty of the disease, its unknown
characteristics, and substantial but varying interventions to reduce its spread.
To alleviate this issue, we propose a new method to disentangle the properties
of the underlying contagion process from its concrete realizations across
spatial entities. Our aim is to separate region-specific aspects -- such as
demographics, enacted policies, and testing methods -- from disease-inherent
aspects that influence its spread. This allows us to train high-resolution
models which jointly model the spread and are able to borrow strength across
regions. In our experiments, we demonstrate that our approach achieves
state-of-the-art performance in predicting the spread of the COVID-19 and
improves the robustness of forecasts.
\end{abstract}

\section{Introduction}
\label{sec:org25bfe58}
\begin{itemize}
\item Forecasting the spread the of COVID-19 is an important task in the public
health response to the disease. Not only important to understand the progress
of the disease, but also central to efficiently allocate scarce resources such
as ventilators, personal protective equipment, and ICU beds.
\item For resource allocation, it is especially important to do forecasts with a
high spatial and temporal resolution
\item At the same time, forecasting COVID-19 poses unique challenges
\begin{itemize}
\item Novel disease with unknown characteristics
\item Very few data, especially at the beginning of the spread. However even
several months after the initial outbreak substantially less data than for other
infectious diseases like influence, measles etc.
\item Simultaneous spread in regions with very different properties
\begin{itemize}
\item demographics and population density
\item enacted policies
\item adherence to enacted policies
\item general mobility
\item geographic features such as temperature
\item testing and reporting
\end{itemize}
\end{itemize}
\end{itemize}

\section{Method}
\label{sec:org10c1259}
\begin{itemize}
\item We assume we have a set of timeseries \(\Set{Y} = \{(y_i^1, \ldots,
  y_i^T)\}_{i=1}^M\) which are different realizations of the the same underlying
contagion process.
\end{itemize}

\subsection{Latent Coupling}
\label{sec:org869a27e}
Autoregressive model of order \(p\) AR(p)
\begin{equation}
    y^{t+1} = \sum_{k=0}^p w_k y^{t-k} + \epsilon
\end{equation}

\begin{equation}
\beta_i^t y_i^t \approx \alpha_{ij }\beta_j^t y_j^t
\end{equation}

\subsubsection{Univariate Coupling}
\label{sec:orgd42cd60}

\subsubsection{Multivariate Coupling}
\label{sec:org61f2c7e}

\subsection{Maximum Likelihood Estimation}
\label{sec:org088c534}
The popularity of the NB distribution is due largely to its ability to model
count data with varying degrees of overdispersion. The distribution is commonly
expressed in terms of the mean \(\mu\) and dispersion parameter \(\nu\) such that the
probability of observing a non-negative integer \(y\) is

\begin{equation*}
\Pr(Y = y) = \frac{\Gamma(y + \nu)}{y!\Gamma(\nu)}\left(\frac{\mu}{\mu +\nu}\right)^{y}\left(1 + \frac{\mu}{\nu}\right)^{-\nu}
\quad \mu > 0, \nu > 0
\end{equation*}

\begin{equation*}
    y^{t+1}_{i} \sim \text{NB}(\eta_i^{t}, \nu_i)
\end{equation*}

\section{Related Work}
\label{sec:org513f1fd}

\section{Experiments}
\label{sec:org23dcef3}
\subsection{Main Experiment}
\label{sec:orga5ea523}
\begin{itemize}
\item Forecast Deaths and Confirmed Cases
\item Weekly backfill on US Nytimes data
\item Weekly backfill on Italy/Spain
\item Separate forecast horizons (allows us to also do CV on earlier days)
\begin{itemize}
\item 7 days
\item 14 days
\item 21 days
\end{itemize}
\item Baselines
\begin{itemize}
\item Standard AR model (univariate)
\item Standard AR model (multivariate)
\item S(E)IR
\item MHPs
\item Naive
\end{itemize}
\end{itemize}
\subsection{Ablation}
\label{sec:org3834428}
\begin{itemize}
\item Univariate
\item Multivariate
\item Without Negative Binomial
\item Without Latent Beta
\item Without Coupling
\end{itemize}

\section{Conclusion}
\label{sec:org09e2e2a}
\end{document}
